% APL: Powerful vocabulary
% 4th: Extensible language, simplest possible syntax :-)
\documentclass{beamer}
\usepackage{beamerthemesplit, eso-pic, graphics, german}
\usetheme{VAXMAN}
%
\newcommand*{\NIX}{\vspace*{.3cm}\\}
\newcommand*{\F}{{\tt\bf lang5}}
%
\title{\F}
\author{Dr. Bernd Ulmann}
\date{26-SEP-2020}
%
\begin{document}
 \begin{frame}[containsverbatim]
  \titlepage
 \end{frame}
%
 \section{Introduction}
  \begin{frame}
   \frametitle{Introduction}
   The following talk is about a new (rather eclectic) interpretive programming
   language called \F{ } which is, in essence, the result of combining the
   basic ideas and aspects of Forth and of APL.
   \NIX
   The idea of extending Forth is not new to say the least -- the most 
   prominent example being HP's well known programming language RPL which
   was for many years the work horse of their top-of-the-line pocket
   calculators like the HP-28, the HP-48 etc.
   \NIX
   While RPL, short for \emph{Reverse Polish LISP},{~} combines Forth's stack
   approach to programming with typical list processing operations normally
   to be found in the language LISP, the language described briefly in the
   following is also based on Forth but extends it with the programming
   paradigm of APL, Ken Iverson's brain child which is still unmatched when
   it comes to elegance and compactness of code.
  \end{frame}
%
  \begin{frame}
   \frametitle{Why \F}
   Why would one create yet another programming language? Aren't there
   more than enough already? The main reasons for the development described
   in the following are these:
   \begin{itemize}
    \item Writing compilers and interpreters is really interesting and gives
     an insight into the design of programming languages which is hard to
     achieve otherwise.
    \item Both Forth and APL have features making them more or less unique
     in the programming languages zoo, so both languages are definitely 
     worth to be at least taken into account as the basis for new developments.
    \item On the other hand, both languages have their deficiencies like 
     the need for special characters in APL, the non-overloading of basic
     arithmetic operators in Forth etc.
    \item Languages based on array operations like APL might be an ideal tool
     to phrase algorithms for vector processors and GPUs. It should be
     worthwhile to think about a compiler generating CUDA-code from \F-programs.
   \end{itemize}
  \end{frame}
%
 \section{\F{ } at a glance}
  \begin{frame}
   \frametitle{\F{ } at a glance}
   What are the main characteristics of \F?
   \begin{itemize}
    \item \F{ } is completely stack based, its builtin operations are called
     \emph{built in words}{ } while user defined operations are 
     just called \emph{words}.
    \item The stack can hold entities of arbitrary structure as long as these
     may be represented as nested arrays.
    \item All built in words as well as user defined words operate on 
     those arbitrary structures which can be placed on the stack. Thus
     {\tt 2 3 + .} yields {\tt 5}{ } while 
     {\tt [1 2] [3 4]{ } + .} yields {\tt [4 6]}.
    \item During startup the \F-interpreter looks for a file named
     {\tt stdlib.5} -- if one is present it will be loaded prior to loading
     the program to be executed. This standard library contains \F-extensions
     written in \F{ } (for example the words {\tt grot} and {\tt ggrot}).
   \end{itemize}
  \end{frame}
%
 \section{\F{ } implementation}
  \begin{frame}[containsverbatim]
   \frametitle{The implementation of \F}
   The following slides give a short overview of the current implementation 
   of \F.
   \NIX
   This implementation is neither complete nor stable enough to be called
   production grade -- its main purpose is to serve as a proof of concept
   of the language itself and its basic implementation concepts.
   \NIX
   The current \F-implementation is based on Perl which led to a very
   rapid development of the interpreter which took, until now, only about two
   man weeks.
   \NIX
   A typical \F-program could look like this:
   \begin{verbatim}
4 iota dup '* outer 
dup 2 compress .
   \end{verbatim}
  \end{frame}
%
  \begin{frame}
   \frametitle{Executing \F}
   Such a \F-program is rather simple to scan, parse and execute:
   \begin{itemize}
    \item A \F-program is parsed by basically splitting its source on 
     whitespace with some special provision for arrays. The basic entities of 
     a \F-program are called \emph{tokens}.
    \item Word definitions start with {\tt :}{ } and end with {\tt ;}. 
     The start of a word definition has highest priority for the interpreter.
    \item If no word is to be defined, \F{ } tries to execute a built in 
     word of the name found in the current token.
    \item If no matching word can be found \F{ } tries to execute a user
     defined word named like the token read.
    \item If no word is found, \F{ } checks if there is a variable with a 
     matching name -- if it succeeds the contents of that variable will just be
     pushed onto the stack.
    \item If even this did not work the element just read will be pushed onto
     the stack.
   \end{itemize}
  \end{frame}
%
  \begin{frame}
   \frametitle{Scanning and parsing}
   If there were no nested structures to be pushed onto the stack and no
   {\tt if}-{\tt else}-constructions or loops, scanning and parsing
   \F{ } would be really trivial.
   \NIX
   To handle nested data structures like {\tt [1 [2 3]{ } 4]} special treatment
   of the tokens generated by splitting the source code at whitespaces
   is needed since these tokens would look like 
   {\tt [1}, {\tt [2}, {\tt 3]}{ } and {\tt 4]}{ }
   which does not represent what the programmer intended.
   \NIX
   Therefore the raw program representation is subjected to a special step 
   which gathers data of nested data structures and transforms
   the example given above back to {\tt [1 [2 3] 4]} from the token stream.
   \NIX
   The same holds true for nested program structures like 
   {\tt if}--{\tt else}--{\tt then}- and 
   {\tt do}--{\tt loop}-structures
   which are processed similarly and yield a nested program structure for
   every controlled block.
  \end{frame}
%
  \begin{frame}
   \frametitle{Data structures}
   Thus a \F-program is represented within the interpreter as a nested array
   containing 
   \begin{itemize}
    \item an entry for every word to be executed,
    \item an entry for each scalar used in the program,
    \item an entry containing a reference to a nested array structure for
     every such structure found in the \F{ } source code,
    \item an entry containing a reference to a nested structure for every
     {\tt if-else-then} or {\tt do-loop} controlled block.
   \end{itemize}

   The following example shows a simple \F-program consisting of two nested
   loops and its internal representation in the interpreter.
  \end{frame}
%
  \begin{frame}[containsverbatim]
   \frametitle{Example structure}
   {\small 
    \begin{verbatim}
0 do
    100 do
        dup .
        1 +
        dup 105 > if
            break
        then
    loop
    drop
#
    dup .
    1 +
    dup 5 > if
        break
    then
loop
drop
    \end{verbatim}
   }
  \end{frame}
%
  \begin{frame}[containsverbatim]
   \frametitle{Example structure}
   The preceding program is then represented internally like this:
   {\small 
    \begin{verbatim}
[ '0', 'do',
  [ '100', 'do',
    [ 'dup', '.', '1', '+', 'dup', '105', '>', 'if',
      [ 'break' ]
    ],
    'drop', 'dup', '.', '1', '+', 'dup', '5', '>', 'if',
    [ 'break' ]
  ],
  'drop'
]
    \end{verbatim}
   }
  \end{frame}
%
 \section{Language elements}
  \begin{frame}[containsverbatim]
   \frametitle{Language elements}
   Currently the following words are implemented:
   \NIX
   \begin{description}
    \item [Binary built in words:] 
     \begin{verbatim}
+ - * / & | ^ > < == >= <= != <=> 
% ** eq ne gt lt ge le
     \end{verbatim}
    \item [Unary built in words:]
     \begin{verbatim}
not neg ! sin cos ? int
     \end{verbatim}
    \item [Stack operations:]
     \begin{verbatim}
dup drop swap over rot depth
     \end{verbatim}
    \item [Array operations:]
     \begin{verbatim}
iota reduce remove outer in select 
expand compress reverse
     \end{verbatim}
    \item [IO-operations:]
     \begin{verbatim}
. .s .v read
     \end{verbatim}
    \item [Variable operations:]
     \begin{verbatim}
set del
     \end{verbatim}
    \item [Control operations:]
     \begin{verbatim}
if else then do loop break exit
     \end{verbatim}
   \end{description}
  \end{frame}
%
 \section{Examples}
  \begin{frame}
   \frametitle{Examples}
   Although the \F-interpreter is far from being complete, some more or less
   simple and actually working examples can already be shown:
   \NIX
   These examples include
   \begin{itemize}
    \item Some introductory programs, 
    \item cosine approximation using MacLaurin series and 
    \item the sieve of Eratosthenes.
   \end{itemize}
  \end{frame}
%
  \begin{frame}
   \frametitle{Fibonacci numbers}
   Recursion is such a powerful tool that not only the \F-interpreter itself
   is highly recursive internally but the language \F{ } itself allows 
   recursion as well.
   \NIX
   The following program computes the well known Fibonacci number sequence 
   implementing the recursive definition
   \begin{eqnarray}
    f(0)&=&1\nonumber\\
    f(1)&=&1\nonumber\\
    f(n)&=&f(n-1)+f(n-2).\nonumber
   \end{eqnarray}
  \end{frame}
%
  \begin{frame}[containsverbatim]
   \frametitle{Fibonacci numbers}
   \begin{verbatim}
: fib 
    dup 2 <
    if
        drop 1
    else
        dup
        1 - fib
        swap 2 - fib
        +
    then
;

0 do
    dup fib . 1 + 
    dup 10 > if break then
loop
   \end{verbatim}
  \end{frame}
%
  \begin{frame}[containsverbatim]
   \frametitle{Throwing dice}
   Throw a dice 100 times and calculate the arithmetic mean of the results:
   \begin{verbatim}
: throw_dice
    100 dup iota undef ne 6 * 
    ? int 1 + 
    '+ reduce swap / .
;
   \end{verbatim}
  \end{frame}
%
  \begin{frame}
   \frametitle{Cosine approximation}
   The following word computes the cosine of a value given in radians using 
   the MacLaurin expansion
   \begin{displaymath}
    \cos x\approx\sum\limits_{i=0}^n (-1)^i \frac{x^{2i}}{(2i)!}
   \end{displaymath}
   with 9 terms.
   \NIX
   To accomplish this without an explicit loop three basic vectors representing
   $(-1)^i$, $x^{2i}$ and $(2i)!$ are generated. Multiplying the first two and
   dividing the result by the third one yields a vector which is then processed
   by summing all of its elements using the {\tt reduce} operation.
   \NIX
   The following slide shows the complete word definition of {\tt mc\_cos}:
  \end{frame}
%
  \begin{frame}[containsverbatim]
   \frametitle{Cosine approximation}
   {\small
    \begin{verbatim}
: mc_cos 
    # Save x and the number of terms for future use
    'x set 9 'terms set

    # Generate a vector containing x ** (2 * i)
    terms iota dup undef ne x * swap 2 * dup v2i set **

    # Generate the (2 * i)! vector and 
    # divide the previous vector by this
    v2i ! /

    # Generate a vector of the form [1 -1 1 -1 1 ...]
    terms iota 1 + 2 % 2 * 1 -

    # Multiply both vectors and reduce the result by '+'
    * '+ reduce
;
    \end{verbatim}
   }
  \end{frame}
%
  \begin{frame}
   \frametitle{List of primes}
   The following program implements a form of the sieve of Eratosthenes
   which is quite popular in the APL community. The basic ideas for 
   generating a list of primes between 2 and a given value {\tt n} are these:
   \begin{itemize}
    \item Generate a vector {\tt [1, 2, 3, ..., n]}.
    \item Drop the first vector element yielding {\tt [2, 3, 4, ..., n]}.
    \item Compute the outer product of two such vectors yielding a matrix
     like this:
     \begin{displaymath}
      \begin{pmatrix}
       4&6&8&10&\dots\\
       6&9&12&15&\dots\\
       8&12&16&20&\dots\\
       10&15&20&25&\dots\\
       \vdots&\vdots&\vdots&\vdots&\ddots
      \end{pmatrix}
     \end{displaymath}
   \end{itemize}
  \end{frame}
%
  \begin{frame}
   \frametitle{List of primes}
   \begin{itemize}
    \item Obviously this matrix contains everything but prime numbers, so 
     the next step is to determine which number contained in the original
     vector {\tt [2, 3, ..., n]} is \emph{not} contained in this matrix which
     can be done using the set operation {\tt in}. 
    \item The result of {\tt in}{~} is a vector with {\tt n-1} elements 
     each being {\tt 0} (its corresponding vector element was not found in 
     matrix and is thus not prime) or {\tt 1}.
    \item After inverting this binary vector it can be used to select
     all prime numbers from the initial vector {\tt [2, 3, ..., n]}.
   \end{itemize}

   This is accomplished by the following \F-program:
  \end{frame}
%
  \begin{frame}[containsverbatim]
   \frametitle{List of primes}
   \begin{verbatim}
: prime_list
  iota 1 +
  0 remove
  dup dup dup
  '* outer
  swap in not
  select
;

100 prime_list .
   \end{verbatim}

   This program yields the following output:
   \begin{verbatim}
[2 3 5 7 11 13 17 19 23 29 31 37 41 
 43 47 53 59 61 67 71 73 79 83 89 97 ]
   \end{verbatim}
  \end{frame}
%
 \section{Style}
  \begin{frame}[containsverbatim]
   \frametitle{Style}
   Combining the power of Forth and APL, \F{ } requires a consistent 
   programming style and rational factoring of words to ensure code
   maintainability. 
   \NIX
   The cosine example from above could have been written also like this:
   \begin{verbatim}
: mc_cos 
    'x set 9 'terms set
    terms iota dup undef ne x * swap 2 * dup v2i set 
    ** v2i ! / terms iota 1 + 2 % 2 * 1 - * '+ reduce
;
   \end{verbatim}

   This code is not really what one would call maintainable compared with the
   far better formatting and commenting shown in the original example.
  \end{frame}
%
  \begin{frame}
   \frametitle{Style}
   All in all the following topics should be taken into account when programming
   in \F:
   \begin{itemize}
    \item Use short word definitions.
    \item Words should do only one thing.
    \item Words should have no side effects.
    \item Indentation of control and data structures is vital for readability.
    \item Resist the temptation of using really clever programming trickery!
     :-) (It is hard, but\dots)
   \end{itemize}
  \end{frame}
%
 \section{Miscellaneous}
  \begin{frame}
   \frametitle{Miscellaneous}
   \begin{itemize}
    \item Next steps: Add more (complex) words like rho etc.
    \item The source code of the \F-interpreter is available upon request
     and it is planned to setup a Source Forge project for \F.
    \item The power of Perl for implementing interpreters and the like is
     remarkable -- the complete \F-interpreter currently consists of 
     only about 700 lines of code.
    \item I would like to thank Mr. Thomas Kratz for the many hours of peer 
     programming during the implementation of the current \F-interpreter.
    \item The author can be reached at
     \begin{center}
      {\tt\bf ulmann@analogparadigm.com}
     \end{center}
   \end{itemize}
  \end{frame}
\end{document}
